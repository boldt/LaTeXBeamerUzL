\documentclass[a4paper,DIV=calc, oneside]{scrartcl}
\usepackage[no-math]{fontspec} 
\usepackage{xunicode,xltxtra} 
\defaultfontfeatures{Scale=MatchLowercase} 
\usepackage{amsmath,amsthm,euscript} % AMS-LaTeX  
\defaultfontfeatures{Mapping=tex-text} % converts LaTeX specials (``quotes'' --- dashes etc.) to unicode
\setsansfont[Ligatures={Common}]{Myriad Pro}
%\setromanfont[Ligatures={Common}]{MinionPro}
\renewcommand{\familydefault}{\sfdefault}
\usepackage{MnSymbol}

\usepackage{calc,scrpage2}
\usepackage{listings,wrapfig, enumerate}
\usepackage[ngerman]{babel} % Neue Rechtschreibung
\usepackage{graphicx,hyperref,booktabs}
\usepackage[pdftex,dvipsnames]{xcolor}
\definecolor{UzLcolor}{cmyk}{1,.0,.20,.78}
\definecolor{UzLred}{cmyk}{0,.87,.89,.24}

%\colorlet{maincolor}{black}
\colorlet{mixcolor}{black!50}

\lstloadlanguages{TeX}
\lstset{basicstyle={\rmfamily\small},numbers=left,numberstyle=\tiny\color{UzLcolor},numbersep=5pt, breaklines=true, captionpos={t},language={},frame=none,numbers=left,tabsize=3, showspaces=false, showtabs=false, columns=fixed}	
\setlength{\parindent}{1em} 
\setlength{\parskip}{0em}	
\setlength{\marginparwidth}{3cm}
\colorlet{maincolor}{black!80}

\hypersetup{ 
   pdftitle={Karteikarten in LaTeX}%
	  ,pdfauthor={Ronny Bergmann}%
	  ,pdfcreator={LaTeX, hyperref, KOMA-Script, TextMate},
   pdfkeywords={Cards, Karteikarten, TeX, Paket, Anleitung, Manual}, 
   pdfdisplaydoctitle,
   bookmarksnumbered=true,
   bookmarksopen=true,
   bookmarksopenlevel=1,
   plainpages=false,
   bookmarksnumbered,
   pdfborder={0 0 0}
   }

\addtokomafont{sectioning}{\color{UzLcolor}}
\setkomafont{descriptionlabel}{\bfseries\color{UzLcolor}}
\newcommand{\todo}[1]{{\color{red}!}\marginpar{\texttt{\color{red}#1}}}
\newcommand{\missing}{{\color{red}!}}
\newcommand{\missingcmd}[1]{\marginpar{\texttt{#1}\ {\color{red}!}}}
\newcommand{\cmd}[1]{\marginpar{\texttt{#1}}}

\defpagestyle{plain}%
{{\includegraphics[height=2\baselineskip]{logos/unilogo300dpi.png}}{}{}}%
{{\includegraphics[height=2\baselineskip]{logos/unilogo300dpi.png}}{}{}}%



  \setheadwidth[0pt]{textwithmarginpar}
  \setheadsepline[\textwidth]{current}[]
  
  \newcommand{\pagenumboxright}{%
%		\colorbox{UzLcolor!20}{
			\parbox[c][4ex][c]{\marginparwidth}{{\color{gray}$\bigl|$}\hspace{1.5em}\thepage}
%		}
	}
	  \newcommand{\pagenumboxleft}{%
	%		\colorbox{UzLcolor!20}{
				\parbox[c][4ex][c]{\marginparwidth}{\thepage\hspace{1.5em}{\color{gray}$\bigl|$}}
	%		}
		}


  \newcommand{\headerleftbox}{
%		\colorbox{maincolor!20}{
			\parbox[c][4ex][c]{\textwidth}{\quad\displaytitle\quad\quad\leftmark}
%		}
	}
  \newcommand{\headerrightbox}{
%		\colorbox{maincolor!20}{
			\parbox[c][4ex][c]{\textwidth}{\quad\rightmark}
%		}
	}
  \renewcommand{\headfont}{%
		%\color{maincolor}%
		\footnotesize}
  
  % left header
  \newcommand{\myleftheader}{%
    \setlength{\fboxsep}{0pt}%
    \pagenumboxleft%
    \hfill\headerleftbox%
  }

  % right header
  \newcommand{\myrightheader}{%
    \setlength{\fboxsep}{0pt}%
    \headerrightbox\hfill\pagenumboxright%
  }

  % left footer
  \newcommand{\myleftfooter}{\normalfont%\thepage
	\hfill}

  % right footer
  \newcommand{\myrightfooter}{\hfill\normalfont%\thepage
			}

% defining pagestyle
  \defpagestyle{mypagestyle}%
    {{\myleftheader}{\myrightheader}{\myrightheader}}%
    {{\myleftfooter}{\myrightfooter}{\myrightfooter}}%



  \pagestyle{mypagestyle}

%
%
% Dokumentanfang
%
%

\begin{document}
	\thispagestyle{empty}
\title{Ein \LaTeX-Beamer-Theme im Corporate-Design der Universität zu Lübeck\\{\large\normalfont Version 0.8}}
\author{Ronny Bergmann\\\emph{bergmann@math.uni-luebeck.de}}
\date{Januar 2010}
\maketitle
\section{Einleitung}
Seit der Einführung des Corporate Design für die Universität zu Lübeck im April 2009 sind viele Schritte zu einem einheitlichen Auftreten der Universität unternommen worden. Dieses \LaTeX-Beamer-Theme überträgt nun die Ideen der
Powerpoint-Vorlage auf \LaTeX. Dabei ist der Standard so orientiert, dass exakt die Folien wie in der Vorlage entstehen.

Zusätzlich gibt es viele Optionen, die eine Anpassung ermöglichen, etwa bezüglich weiterer Logo-Einblendunden oder
einer Fußzeile mit Folienzahlen, die in \LaTeX\ häufig Verwendung findet. Die Einstellungen können direkt beim Einbinden als Optionen gegeben werden, aber auch später neu gesetzt werden.

\section{Paketoptionen}
Eine Übersicht mit Stichpunktartiger Erklärung findet sich in Tabelle \ref{tab:Paketoptionen}
\begin{table}
	\begin{tabular}{llll}
		\toprule
		\textbf{Paketoption} & \textbf{Beschreibung}\\\midrule
		\lstinline!footline=true|false! & Anzeige der Fußzeile an bzw. \textbf{aus}\\
		\lstinline!logos=1|2|3! \missing & Anzahl zu definierender Logos, Standard: 1\\
		\lstinline!displayNaviagation=true|false! & Navigations-Zeile neben dem Logo \textbf{an} bzw. aus.\\
		\lstinline!displayTotalFrameNumber=true|false! & Anzeige der Gesamtfolienanzahl \textbf{an} bzw. aus\\
		\lstinline!slogan=true|false! & Anzeige des Slogans „Im Focus das Leben“ \textbf{an} bzw. aus\\
		\lstinline!useMyriad=true|false! & Verwendung der Hausschriftart Myriad Pro \textbf{an} bzw. aus\\\bottomrule
	\end{tabular}
	\caption{Paketoptionen, die bei Aktivierung des Themes mitgegeben werden können}
	\label{tab:Paketoptionen}
\end{table}
\section{Befehle}
\missingcmd{setlogo}
\missingcmd{setlogowidth}
\section{mögliche weitere Features}
\section{bekannte Probleme}
Die Verwendung der Hausschriftart Myriad Pro stellt sich unter \LaTeX\ als eine Herausforderung heraus. Momentan funktioniert das Paket nur im Zusammenspiel mit dem etwas neueren XeTeX-Compiler, eine Anpassung auch an \lstinline!pdflatex! wäre aufgrund der Kompatibillität wünschenswert. Auf die Installation der Schrift soll hier nicht eingegangen werden.
\section{weitere Quellen des Paketes}
Git-Adresse
Adresse auf der HP
\newpage
\end{document}