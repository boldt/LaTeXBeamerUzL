\documentclass[german,10pt,xcolor=colortbl]{beamer}
\usepackage[utf8]{inputenc}
\usepackage[OT1]{fontenc}
\usepackage{mathptmx,courier,calc}
\usepackage[scaled]{helvet}
\usepackage[ngerman]{babel} % Neue Rechtschreibung
\usepackage{amsmath,amsthm,amssymb,euscript} % AMS-LaTeX  
\usepackage{listings,wrapfig, enumerate}
%
\usepackage{acronym}
\usepackage{colortbl,tabularx,graphicx}

\usetheme[footline=false]{UzL}
%\usecolortheme[RGB={0,75,90}]{structure} 

\setbeamertemplate{navigation symbols}{}
\title{Multivariate translationsinvariante Räume}
\date[]{4. Februar 2011\\[1ex] 21. Rhein-Ruhr-Workshop in Königswinter}
\author[R. Bergmann]{Dipl.-Inf. R. Bergmann}
\institute[Universität zu Lübeck]{Institut für Mathematik\\Universität zu Lübeck}
\begin{document}

	\begin{frame}[plain]
		\leavevmode
		\vskip-.2ex
		\hskip-1.1em
	  \pgfuseimage{uzllogo}
		\vskip.5ex
  	  \begin{beamercolorbox}[wd=.95\paperwidth, ht=.853\paperheight,leftskip=.3cm,rightskip=.3cm plus1fil,vmode]{frametitle}
			\centering
		    {\usebeamerfont*{frametitle}\inserttitle}%
		    \vskip6ex
			{\usebeamerfont*{text}\color{black}
			\insertauthor
			\vskip6ex
			}
			{\usebeamerfont*{text}
			\small
			\color{black}\insertinstitute
			\vskip6ex
			}
			{\usebeamerfont*{text}\usebeamercolor{text}
				\insertdate}
			\vskip4ex  
			\vfill
			\vskip2ex
			\hfill\hskip-1.8em\pgfuseimage{uzlslogan}
			\begin{beamercolorbox}[wd=.9em]{frametitle}
			\end{beamercolorbox}
			\vskip.7\baselineskip
	  \end{beamercolorbox}%
	\end{frame}
	
	\begin{frame}{Inhalt}
		\tableofcontents%[sections=<1-4>]
	\end{frame}
	\section{Muster}
	\begin{frame}{Title}{Subtitle}
		ABC und hier noch ein wenig text bis er umbricht und dann schauen wir mal was so passiert.
	\end{frame}
	
	\section{Translationsinvarianz}
	\begin{frame}{Title}{Subtitle}
		\begin{example}[ABC]
			CDExample
		\end{example}
		\begin{block}[ABC]
			block
		\end{block}
		\begin{definition}[Bin ich]
			auch
		\end{definition}
	\end{frame}
	\begin{frame}{Title}{Subtitle}
		\begin{alert}[ABC]
			asfg
		\end{alert}
	\end{frame}
	\section{Inklusionsbeziehung}
	\begin{frame}{Title}{Subtitle}
		ABC
	\end{frame}
	\begin{frame}{Title}{Subtitle}
		ABC
	\end{frame}
	\begin{frame}{Title}{Subtitle}
		ABC
	\end{frame}

\end{document}
