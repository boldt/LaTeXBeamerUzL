\documentclass[german,10pt,xcolor=colortbl
%,draft
]{beamer}
\usepackage{xunicode} 
\usepackage[OT1]{fontenc}
\usepackage{calc}
\usepackage[ngerman]{babel} % Neue Rechtschreibung
\usepackage{amsmath,amsthm,amssymb,euscript} % AMS-LaTeX  
\usepackage{enumerate,graphicx}

% Load Theme
\usetheme[footline=true, slogan=false]{UzL}
%
\setbeamertemplate{navigation symbols}{}
\title{Beispielpräsentation}
\subtitle{Eines Vortrages im neuen Design mit \LaTeX}
\date[]{5. Februar 2011\\[1ex] Workshop on Corporate Design Stuff}
\author[R. Bergmann]{Ronny Bergmann}
\institute[Universität zu Lübeck]{Institut für Mathematik\\Universität zu Lübeck}
\begin{document}

	\begin{frame}
		\titlepage
	\end{frame}
	
	\section{Einleitung}
	\begin{frame}{Inhalt}
		\tableofcontents%[sections=<1-4>]
	\end{frame}
	
	\begin{frame}{Einleitung - Motivation}
	ABC
	\begin{lemma}[Ein Beispiellemma]
		Ist das hier und es gilt.
	\end{lemma}
	\begin{example}
		Ein Beispiel, wie dieses hier
	\end{example}
	\alert{ACHTUNG!}
		Etwas Hervorgehobenes. Die Farben sind bisher alle dem Handbuch zum Corporate Design entnommen.
	\end{frame}
	\begin{frame}{Further}{Ein Untertitel auf deutsch.}
			\begin{proof}
				Weil.
			\end{proof}
	\end{frame}
	\begin{frame}{TODO}
		\begin{itemize}
			\item Literatur?
		\end{itemize}
	\end{frame}
\end{document}
